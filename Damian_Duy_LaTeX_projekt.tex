\documentclass[12pt]{article}
\usepackage[T1]{fontenc}
\usepackage[utf8]{inputenc}
\usepackage[polish]{babel}
\usepackage{graphicx}
\usepackage{amsmath}
\usepackage{refstyle}
\usepackage{subfig}
\selectlanguage{polish}
\title{Różności Matematyczne}
\author{Damian Duy}
\date{10.01.2020}
\begin{document}
\maketitle
\noindent
W poniższej pracy zostanie omówiona część ważnych zagadnień matematycznych z różnych poziomów edukacji.
\section{Bryły}
Poniżej znajduje się lista najważniejszych brył:
\begin{center}
\textit{Uwaga! Należy pamiętać, że ta lista nie jest pełna.}
\end{center}
\begin{itemize}
\item Stożek - jaki jest każdy wie, a jak ktoś nie wie to poniżej będą zdjęcia.
\item Piramida - ważna bryła, na jej podstawie powstały piramidy w Egipcie.
\item Czworościan foremny - dużo z jego ścian to trójkąty.
\item Ośmiościan foremny - ma zadziwiająco nieskomplikowany wzór.
\end{itemize}
\subsection{Wzory na objętość i pole powierzchni brył.}
\begin{table}[h]
\caption{Wzór na objętości brył}
\centering
\label{tab:Objbryl}
\scalebox{0.8}{
\begin{tabular}[h]{|c|c|c|}
\hline
Bryła & Wzór na objętość & Wzór na pole powierzchni\\
\hline
Stożek & $\frac{1}{3} \pi r^2h$ & $\pi r^2 + \pi rl$ \\
\hline
Piramida & $\frac{a^2h}{3}$  &$a^2 + 2ah$\\
\hline
Czworościan foremny & $\frac{1}{12} a^3 \sqrt{2}$  & $ a^2\sqrt{3} $\\
\hline
Ośmiościan foremny & $\frac{1}{3} a^3 \sqrt{2}$ &$2 \sqrt{3} a^2$\\
\hline
\end{tabular}}
\end{table}
\newpage
\subsection{Obrazki brył.}
Jak pokazano w \tabref{Objbryl} objętość bryły zależy od róznych czynników. Rzeczą, która pomoże to zoobrazować będą obrazki. \newline
\begin{center}
Cytat nieznanego autora:
\end{center}
\begin{quotation}
\textbf{Na obrazkach najwięcej się nauczysz. One są kluczem do sukcesu. Z nich czerpie się wiedzę.}
\end{quotation}
\begin{figure}[h]
\begin{minipage}[c]{0.3\linewidth}
\includegraphics[width=\linewidth]{rysunek375.jpg}
\caption{Stożek}
\end{minipage}
\hfill %rozciąga poziomo, bez tego byłyby obok siebie
\begin{minipage}[c]{0.3\linewidth}
\includegraphics[width=\linewidth]{rysunek369.jpg}
\caption{Piramida}
\end{minipage}
\end{figure}
\begin{figure}
\begin{minipage}[c]{0.3\linewidth}
\includegraphics[width=\linewidth]{rysunek367.jpg}
\caption{Czworościan foremny}
\end{minipage}
\hspace{5cm}%rozciąga poziomo o 5 cm (przesuwa drugi rysunek)
\begin{minipage}[c]{0.4\linewidth}
\includegraphics[width=0.6\linewidth]{rysunek371.jpg}
\caption{Ośmiościan szcześcienny}
\end{minipage}
\end{figure}
\newpage
\section{Całki}
\end{document}
