\documentclass[12pt]{article}
\usepackage[T1]{fontenc}
\usepackage[utf8]{inputenc}
\usepackage[polish]{babel}
\usepackage{graphicx}
\usepackage{amsmath}
\usepackage{refstyle}
\usepackage{subfig}
\usepackage{url}
\selectlanguage{polish}
\title{Różności Matematyczne}
\author{Damian Duy}
\date{10.01.2020}
\begin{document}
\maketitle
\noindent
W poniższej pracy zostanie omówiona część ważnych zagadnień matematycznych z różnych poziomów edukacji.
\newpage
\tableofcontents
\listoffigures
\listoftables
\newpage
\section{Bryły}
Poniżej znajduje się lista najważniejszych brył:
\begin{center}
\textit{Uwaga! Należy pamiętać, że ta lista nie jest pełna.}
\end{center}
\begin{itemize}
\item Stożek - jaki jest każdy wie, a jak ktoś nie wie to poniżej będą zdjęcia.
\item Piramida - ważna bryła, na jej podstawie powstały piramidy w Egipcie.
\item Czworościan foremny - dużo z jego ścian to trójkąty.
\item Ośmiościan foremny - ma zadziwiająco nieskomplikowany wzór.
\end{itemize}
\subsection{Wzory na objętość i pole powierzchni brył.}
\begin{table}[h]
\caption{Wzór na objętości brył}
\centering
\label{tab:Objbryl}
\scalebox{0.8}{
\begin{tabular}[h]{|c|c|c|}
\hline
Bryła & Wzór na objętość & Wzór na pole powierzchni\\
\hline
Stożek & $\frac{1}{3} \pi r^2h$ & $\pi r^2 + \pi rl$ \\
\hline
Piramida & $\frac{a^2h}{3}$  &$a^2 + 2ah$\\
\hline
Czworościan foremny & $\frac{1}{12} a^3 \sqrt{2}$  & $ a^2\sqrt{3} $\\
\hline
Ośmiościan foremny & $\frac{1}{3} a^3 \sqrt{2}$ &$2 \sqrt{3} a^2$\\
\hline
\end{tabular}}
\end{table}
\subsection{Obrazki brył.}
Jak pokazano w \tabref{Objbryl} objętość bryły zależy od róznych czynników. Rzeczą, która pomoże to zoobrazować będą obrazki. \newline
\begin{center}
Cytat nieznanego autora:
\end{center}
\begin{quotation}
\textbf{Na obrazkach najwięcej się nauczysz. One są kluczem do sukcesu. Z nich czerpie się wiedzę.}
\end{quotation}
\begin{figure}[h]
\begin{minipage}[c]{0.3\linewidth}
\includegraphics[width=\linewidth]{rysunek375.jpg}
\caption{Stożek \cite{MediaNauka}}
\end{minipage}
\hfill %rozciąga poziomo, bez tego byłyby obok siebie
\begin{minipage}[c]{0.3\linewidth}
\includegraphics[width=\linewidth]{rysunek369.jpg}
\caption{Piramida \cite{MediaNauka}}
\end{minipage}
\end{figure}
\begin{figure}
\begin{minipage}[c]{0.3\linewidth}
\includegraphics[width=\linewidth]{rysunek367.jpg}
\caption{Czworościan foremny \cite{MediaNauka}}
\end{minipage}
\hspace{5cm}%rozciąga poziomo o 5 cm (przesuwa drugi rysunek)
\begin{minipage}[c]{0.4\linewidth}
\includegraphics[width=0.6\linewidth]{rysunek371.jpg}
\caption{Ośmiościan szcześcienny \cite{MediaNauka}}
\end{minipage}
\end{figure}
\newpage
\section{Macierze}
Działania jakie możemy wykonywać na macierzach:
\begin{enumerate}
\item dodawanie
\item odejmowanie
\item mnożenie
\item mnożenie przez skalar
\item odwracanie
\end{enumerate}
\center\emph{Przykład dodawania macierzy}
\cite{MatmaNa6}
\begin{equation}
\label{eq:array1}
\left[ \begin{array}{ccc}
1 & 3 & 2 \\
-1 & 2 & 6 \\
4 & 6 & -8
\end{array} \right]
+
\left[ \begin{array}{ccc}
3 & 9 & 8 \\
1 & 0 & -1 \\
2 & 7 & 8
\end{array} \right]
=
\left[ \begin{array}{ccc}
1+3 & 3+9 & 2+8 \\
-1+1 & 2+0 & 6-1 \\
4+2 & 6+7 & -8+8
\end{array} \right]
\end{equation}
\section{Funkcje}
Funkcje są opisywane wzorami. Znając wzór funkcji można zaznaczyć na układzie współrzędnych punkty, które do niej należą.
\newline
\center\emph{Przykład opisania funkcji wzorem:}
\begin{equation}
\label{eq:function1}
f(x) = \left\lbrace
\begin{array}{ccc}
\frac{-\tan x}{\cos^3 x} & \text{dla} & x < 0,
\\x^5-0.8 & \text{dla} & x = 0,
\\\arcsin x^4 & \text{dla} & x > 0
\end{array}
\right.
\end{equation}
\subsection{Wzory niektórych znanych funkcji}
\begin{table}[h]
\caption{Wzory wybranych funkcji \cite{Edukator}}
\centering
\label{tab:WzFunkcji}
\scalebox{1.1}{
\begin{tabular}[h]{|c|c|}
\hline
Funkcja & Wzór funkcji\\
\hline
Liniowa & $f(x) = ax + b$\\
\hline
Kwadratowa & $f(x) = ax^2 + bx + c$\\
\hline
Logarytmiczna & $f(x) = log_ax$\\
\hline
Wymierna & $f(x) = \frac{1}{x}$\\
\hline
\end{tabular}}
\end{table}
\begin{thebibliography}{9}
\bibitem{MediaNauka}
Media Nauka,
\emph{Objętość}.
\url{https://www.medianauka.pl/objetosc},
ART-1409,
2011.
\bibitem{MatmaNa6}
Matma na 6,
\emph{Działania na macierzach}.
\url{https://www.matmana6.pl/dzialania-na-macierzach},
\bibitem{Edukator}
Edukator,
\emph{Funkcja logarytmiczna}.
\url{https://www.edukator.pl/funkcja-logarytmiczna,745.html},
\end{thebibliography}
\end{document}
